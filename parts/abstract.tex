\chapter*{Abstract}

%The development of fast response semiconductor sensor for particle tracking at LHC has increased in the past decade in response to the high luminosity expected for the upgraded Large Hadron Collider (LHC).

The significant increase of the beam luminosity by a factor three \cite{tdr} is one of the main experimental challenges for the High Luminosity LHC (HL-LHC) physics program, and a new way to mitigate the effects of pile-up is to use high-precision timing information to distinguish between collisions occurring close in space but well-separated in time.

To address this experimental challenge, a High-Granularity Timing Detector (HGTD), based on low gain avalanche detector technology is proposed for the ATLAS Phase-II upgrade. The detector will be installed at the forward region, covering the pseudorapidity range between $2.4< |\eta| <4.0$, and allowing the measurement of the time interval of the order of 20-30 picoseconds.

%The high-precision timing information will greatly improve the track-to-vertex association, leading to a significant increase in performance allowing for a high-precision reconstruction of particle jets. These improvements in jet reconstruction performance will translate into important sensitivity gains and enhance the reach of the ATLAS physics program.

Therefore, in this project, it is proposed to be performed activities on research and development at São Paulo University in collaboration with ATLAS experiment. The research will be focused on the development of the LGDA sensors - recently developed to operate at higher levels of radiation dose - in order to optimize the device capabilities to operate in the ATLAS experiment. %The consolidation of this type of sensors offers a unique opportunity for research and development regarding the semiconductor sensors, with a great spin-off for other areas of physics and general technology.