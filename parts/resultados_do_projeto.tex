% Conteúdo do capitulo 
% 1- Desenvolver e caracterizar sensores LGAD para o upgrade do ATLAS
%2- Estudar experimentalmente os processos físicos relacionados com a amplificação da carga
%3- Melhorar os sensor por intermédio de simulações comparando com dados experimentais
%4- Desenvolvimento de um sistema de aquisição
%5- perspectivas de outros trabalhos
\chapter{Objetivos do projeto}

% 1- Desenvolver e caracterizar sensores LGAD para o upgrade do ATLAS
O objetivo deste projeto é caracterizar e desenvolver sensores semicondutores do tipo LGAD para o upgrade do experimento ATLAS, bem como para outras aplicações no âmbito do Instituto de Física da USP. O trabalho será dividido em fases as quais visam em um primeiro momento consolidar o {\it know how} sobre essa tecnologia no HEPIC, e em um segundo momento, com as competências consolidadas, contribuir para a melhoria do sensor e suas aplicações em diversas áreas da ciência, bem como para o ATLAS.

Como exposto anteriormente, no contexto da colaboração ATLAS, o pesquisador responsável participará ativamente na qualificação dos sensores e na consolidação de suas especificações. Em seguida, uma vez definidos os sensores, os mesmos serão produzidos em larga escala, e neste ponto o grupo fará parte do esforço em conjunto com o experimento ATLAS para qualificar os LGAD que serão empregados na construção do HGTD. Esse trabalho será fundamental para o exercício dos métodos a serem desenvolvidos bem como para a consolidação da importância do grupo de pesquisa HEPIC na esfera da colaboração ATLAS.

%2- Estudar experimentalmente os processos físicos relacionados com a amplificação da carga
Em seguida, com a implantação das diversas metodologias experimentais para a caracterização dos sensores semicondutores será possível estudar em grande detalhe os processos físicos relacionados com a amplificação de carga e a produção de sinal no material semicondutor, visando compreender e melhorar os processos de fabricação baseando-se no aumento do ganho, resolução temporal e estabilidade elétrica dos sensores durante sua operação em ambiente com alta radiação, i.e. 4.2MGy \cite{tdr}. 
Como resultado espera-se produzir novas gerações de sensores os quais terão uma excelente performance, podendo ser utilizados não apenas para a detecção de partículas carregadas, mas como detector de raios-X. Devido às excelentes características dos detectores do tipo LGAD, as quais incluem sua alta eficiência quântica para uma grande faixa de comprimentos de onda e a possibilidade de construir detectores com alta granularidade, aplicações em diversos ramos envolvendo a detecção de raios-X, tais como luz síncrotron, tornam-se muito atrativas e de fácil implantação uma vez estabelecida essa tecnologia. Essa também é uma das propostas do projeto para longo prazo.

%3- Melhorar os sensor por intermédio de simulações comparando com dados experimentais
Outro aspecto importante que também será trabalhado neste projeto é o estudo de melhorias na geometria do LGAD por intermédio de simulações e modelos computacionais, e a comparação direta com dados experimentais medidos em laboratório para a validação do modelo. Esse trabalho será fundamental para possibilitar a melhoria do sensor, além de produzir um modelo validado capaz de fazer predições para o caso de novos designs de sensores semicondutores. 

%4- Desenvolvimento de um sistema de aquisição
Além do desenvolvimento dos sensores LGAD, um outro objetivo deste projeto será o de desenvolver um sistema de aquisição de dados capaz de ser integrado ao sensor o que tornará possível a reconstrução de eventos. Como resultado espera-se desenvolver todas as competência e habilidades presentes nos diversos componentes que compõem o sistema de aquisição, desde os aspectos físicos relacionados com a produção do sinal até o tratamento dos dados e imagens produzidas, tornando-se desse modo um conhecimento adquirido importante para o grupo HEPIC tanto quanto para o Departamento de Física Nuclear do Instituto de Física da USP.

%5- perspectivas de outros trabalhos
%Por fim, o desenvolvimento de algorítimos para a reconstrução dos dados do experimento ATLAS utilizando o HGTD em conjunto com o ITk não serão inclusos de início como objetivos neste projeto, no entanto tendo em vista a importância deste componente para o desenvolvimento do sistema de detecção e aquisição o mesmo poderá, como uma perspectiva futura, ser trabalhado mais adiante no projeto. %dependendo da quantidade de recursos disponíveis para serem realocados para essa atividade.  

% Este projeto tem como objetivo estrategico preparar o terreno para projetos futuros no campo de semicondutores e para a continuacao com projetos tematicos.
\renewcommand{\cleardoublepage}{}
\renewcommand{\clearpage}{}