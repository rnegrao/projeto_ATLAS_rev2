

Eu gostaria de contribuir inicialmente através da promoção de um ambiente propício e aberto a discussão de ideias inovadoras que envolvam física, ciências em geral, engenharia e computação que possam resultar, em um segundo momento, na criação de projetos experimentais, com resultados reais, em diversos campos estratégicos, tais como sensores semicondutores, microeletrônica e instrumentação científica avançada, com grande potencial de impactar a sociedade Brasileira.

Eu acredito que o Insper e um ambiente fértil - pelo sua cultura interdisciplinar e multicultural - para a criação de um ambiente de inovação com um potencial ilimitado para a geração de conhecimento sobretudo nas engenharias.


Possuí graduação em Física pela Universidade Estadual de Maringá (2005), mestrado em física nuclear pela Universidade de São Paulo (2009) e doutorado em física nuclear de altas energias pela Universidade de São Paulo (2014). 

Possuí experiência profissional em Física Nuclear Experimental com trabalhos realizados no Brookhaven National Laboratory (BNL), como pesquisador visitante, e no European Organization for Nuclear Research (CERN), como Pos- doutor.