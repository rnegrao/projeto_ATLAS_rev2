\chapter*{Resumo}

%A necessidade de desenvolver detectores semicondutores rápidos para a medida de trajetória de partículas carregadas no {\it Large Hadron Collider} (LHC) tem aumentado no decorrer da última década em resposta ao aumento da luminosidade do feixe produzido no LHC.

Com o crescente aumento da luminosidade do feixe produzido no LHC, que será três vezes maior após o seu upgrade \cite{HL_LHC,tdr}, inúmeros desafios experimentais são colocados, dentre eles está a identificação do grande número de colisões que ocorrem durante o cruzamento do feixe no LHC. A vista disso, uma forma de abordar os efeitos do empilhamento de colisões, e mitigar esse problema, é através do uso de técnicas de medida de tempo com alta precisão para distinguir as colisões que não podem ser separadas espacialmente. 

Para solucionar esse problema, um novo sistema de detecção será construído - denominado {\it High Granularity Timing Detector} (HGTD) - para a fase II do experimento ATLAS, o qual é baseado em sensores semicondutores de silício e cuja tecnologia empregada permite operá-los em um regime de baixo ganho de carga ({\it Low Gain Avalanche Detector} - LGAD). Esse detector será instalado na região frontal do experimento, cobrindo o intervalo em pseudo-rapidez de $2.4< |\eta| <4.0$, e permitirá medir intervalos de tempo da ordem de 20-30 pico segundos.

%Por conseguinte, a medida precisa do tempo irá melhorar a reconstrução do vértice da colisão, tendo em vista que diminui as incertezas na associação das trajetórias ao vértice onde foram originadas, possibilitando o aumento significativo do desempenho, e permitindo reconstruir jatos de partículas com grande precisão. A melhoria na capacidade de reconstrução de jatos - que será obtida com o HGTD - aumentará a sensibilidade do experimento permitindo explorar observáveis antes não explorados devido aos limites experimentais.

A vista disso, neste projeto de pós-doutorado é proposto o desenvolvimento de atividades de pesquisa junto ao grupo de pesquisa HEPIC da Universidade de São Paulo em colaboração com o experimento ATLAS do CERN. O trabalho será focado na pesquisa e desenvolvimento dos sensores semicondutores do tipo LGAD - recentemente desenvolvidos visando aplicações que requerem grande tolerância à altos níveis de radiação ionizante \cite{JIN_LGAD,NIMA_LGAD,NIMA_LGAD_I,NIMA_LGAD_II,NIMA_LGAD_III} - com o objetivo de optimizar sua capacidade para utilização no ATLAS. %A consolidação desse tipo de sensores oferece uma grande oportunidade em termos de pesquisa e desenvolvimento na área de sensores semicondutores, com aplicações em diversas áreas da física e tecnologia em geral.

